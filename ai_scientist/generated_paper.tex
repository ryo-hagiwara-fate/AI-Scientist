```latex
\documentclass[11pt]{article}
\usepackage{amsmath, amssymb, graphicx, hyperref, cite}
\usepackage{algorithm, algpseudocode}
\usepackage{caption, subcaption}
\usepackage{booktabs}

\title{Advancements in Deep Unfolding Networks for Image Processing and Computer Vision Applications}
\author{Your Name \\
\texttt{your.email@example.com}}
\date{}

\begin{document}

\maketitle

\begin{abstract}
Deep unfolding networks have emerged as a powerful paradigm for bridging the gap between model-based optimization techniques and data-driven learning approaches. This paper provides a comprehensive overview of recent developments in deep unfolding networks applied to various image processing and computer vision tasks. We summarize and analyze the contributions and methodologies of key research works, including low-light image enhancement, compressive sensing reconstruction, hyperspectral image reconstruction, single image super-resolution, hyperspectral anomaly detection, hybrid beamforming for THz massive MIMO systems, visual image compressive sensing, and ambient backscatter communications. The findings demonstrate significant performance improvements and efficiency gains over traditional methods, showcasing the potential of deep unfolding networks in advancing the state-of-the-art in these domains.
\end{abstract}

\section{Introduction}
Deep unfolding networks have gained considerable attention in recent years for their ability to integrate the interpretability of model-based optimization with the learning capacity of neural networks. By unfolding iterative optimization algorithms into neural network architectures, these methods offer a powerful framework for solving complex problems in image processing and computer vision. This paper reviews significant advancements in deep unfolding networks, highlighting their applications across various domains and the improvements they bring over traditional techniques.

\section{Related Work}
Previous research in image processing and computer vision has extensively explored both model-based and learning-based approaches. Model-based methods, while interpretable, often require manual parameter tuning and suffer from limited adaptability. On the other hand, learning-based methods leverage large datasets to learn complex representations but lack interpretability and generalization. Deep unfolding networks aim to combine the strengths of both approaches, providing a structured yet flexible framework for various tasks.

\section{Methodology}
The methodologies of the reviewed papers share a common theme of unfolding optimization problems into neural networks, but they vary in their specific implementations and target applications. Below, we summarize the key methodologies of the selected papers:

\subsection{URetinex-Net}
URetinex-Net introduces a Retinex-based deep unfolding network for low-light image enhancement. The method decomposes low-light images into reflectance and illumination layers by unfolding an optimization problem into a learnable network. It incorporates data-dependent initialization, high-efficient unfolding optimization, and user-specified illumination enhancement to adaptively fit implicit priors, achieving noise suppression and detail preservation.

\subsection{DPC-DUN}
The Dynamic Path-Controllable Deep Unfolding Network (DPC-DUN) for compressive sensing reconstruction utilizes a path-controllable selector to dynamically choose processing stages based on image content. This approach balances performance and computational complexity by optimizing the computational efficiency of the unfolding process.

\subsection{PADUT}
The Pixel Adaptive Deep Unfolding Transformer (PADUT) for hyperspectral image reconstruction addresses issues like fixed gradient descent steps and inadequate prior modules for 3D HSI. It employs pixel adaptive descent steps, a Non-local Spectral Transformer (NST), and Fast Fourier Transform (FFT) to enhance stage interaction and reconstruction quality.

\subsection{LRR-Net}
LRR-Net integrates low-rank representation (LRR) with deep learning for hyperspectral anomaly detection. It utilizes the alternating direction method of multipliers (ADMM) optimizer to embed the LRR solution as prior knowledge into the network, transforming regularized parameters into trainable ones.

\subsection{AMP-Net}
AMP-Net unfolds the approximate message passing (AMP) algorithm into a network for visual image compressive sensing, incorporating deblocking modules to address blocking artifacts. The joint training of the sampling matrix with other network parameters improves reconstruction performance.

\subsection{ManNet/subManNet}
For THz massive MIMO systems, ManNet and subManNet propose hybrid beamforming designs by factorizing the optimal fully digital beamformer into analog and digital terms. These methods solve least squares problems using a deep unfolding neural network framework, demonstrating enhanced spectral efficiency and reduced complexity.

\subsection{DUNN}
The deep unfolding neural network (DUNN) for ambient backscatter communications combines data-driven and model-driven approaches for joint beamforming and detection. It treats optimization variables as network parameters to address non-convex optimization challenges, achieving superior detection performance.

\section{Experiments}
The experiments conducted in the reviewed papers span various datasets and evaluation metrics, demonstrating the effectiveness and superiority of deep unfolding networks over traditional methods. Below, we summarize the key experimental setups and findings:

\subsection{Low-light Image Enhancement}
URetinex-Net was evaluated on several benchmark datasets, showing significant improvements in both qualitative and quantitative measures compared to state-of-the-art methods.

\subsection{Compressive Sensing Reconstruction}
DPC-DUN demonstrated high flexibility and excellent performance, with dynamic adjustments offering suitable tradeoffs between performance and computational complexity.

\subsection{Hyperspectral Image Reconstruction}
PADUT outperformed existing methods in terms of reconstruction quality for both simulated and real hyperspectral images, showcasing the benefits of pixel adaptive descent steps and the Non-local Spectral Transformer.

\subsection{Hyperspectral Anomaly Detection}
LRR-Net achieved superior performance across eight datasets, highlighting the efficacy of integrating low-rank representation with deep learning for anomaly detection.

\subsection{Visual Image Compressive Sensing}
AMP-Net showed superior reconstruction accuracy, speed, and efficiency, effectively addressing blocking artifacts and improving overall performance.

\subsection{Hybrid Beamforming for THz Massive MIMO Systems}
ManNet and subManNet-based HBF approaches outperformed traditional methods, demonstrating enhanced spectral efficiency, reduced complexity, and faster run times.

\subsection{Ambient Backscatter Communications}
DUNN provided superior detection performance, effectively addressing non-convex optimization challenges in ambient backscatter communication systems.

\section{Results}
The results of the reviewed papers consistently demonstrate the advantages of deep unfolding networks in various image processing and computer vision tasks. These methods offer significant improvements in performance, adaptability, and efficiency compared to traditional approaches, highlighting the potential of deep unfolding networks to advance the state-of-the-art in these domains.

\section{Conclusion}
This paper reviewed recent advancements in deep unfolding networks, summarizing their contributions, methodologies, and findings across various applications. The reviewed works showcase the potential of deep unfolding networks to combine the interpretability of model-based optimization with the learning capacity of neural networks, offering significant improvements in performance and efficiency. Future research directions include exploring new applications, refining network architectures, and further enhancing the adaptability and generalization capabilities of deep unfolding networks.

\section{References}
\bibliographystyle{ieeetr}
\bibliography{references}
\end{document}
```